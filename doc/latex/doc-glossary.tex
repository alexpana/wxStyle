\makeglossaries

\newglossaryentry{aplica'tie}
{
  name = {aplica'tie},
  description = {Orice component'a software care necesit'a interac'tiunea utilizatorului pentru desf'a'surarea ac'tiunii sale}
}

% \newglossaryentry{aplicatie-nativa}
% {
%   name = aplicație nativă,
%   description = Aplicație care rulează deasupra sistemului de operare având acces direct la resursele hardware și interfața sistemului de operare. Aplicațiile native necesită instalare
% }

% \newglossaryentry{aplicatie-web}
% {
%   name = aplicație web,
%   description = O pagină web prezentată (deschisă) de un browser care oferă un serviciu utilizatorului. Aplicațiile web se aseamănă cu aplicațiile native in funcționalitate dar diferă prin modul în care acestea sunt dispuse utilizatorului
% }

% \newglossaryentry{aplicatie-portabila}
% {
%   name = aplicatie portabila,
%   description = O aplicatie nativa care a fost scrisa pentru a putea fi executata pe mai mult de un sistem de operare.
% }