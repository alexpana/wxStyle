\chapter{Introducere - Contextul proiectului}
\pagestyle{headings}

\section{Introducere}

De la 'inceputurile revolutiei digitale 'si p{\ia}n'a 'in anii 1980, interesul principal al dezvoltatorilor de aplica'tii a fost utilizarea eficient'a a celor mai importante resurse hardware: procesorul 'si memoria. Ast'azi, datorit'a costurilor sc'azute ale componentelor hardware 'si cre'sterea accentuat'a a performan'telor sistemelor de calcul personale, ne permitem s'a optimiz'am o nou'a resurs'a, accea a eficien'tei cu care utilizatorul folose'ste o aplica'tie.

\medskip

C{\ia}nd o persoan'a porne'ste o aplica'tie, fie ca folose'ste un calculator personal sau un dispozitiv mobil, acea persoan'a deschide, 'in esen't'a, o conversa'tie. 'In cazul aplica'tiilor de tip linie de comand'a, utilizatorul "vorbe'ste" cu sistemul software prin comenzile pe care le introduce. Sistemul 'in schimb r'aspunde cu mesaje informative legate de starea 'in care se afl'a sau procesul pe care 'il desf'a'soar'a. 'In cazul aplica'tiilor cu interfa't'a grafic'a, conversa'tia se realizeaz'a sub forma metaforei interfe'tei grafice, adic'a prin butoane, meniuri 'si alte elemente grafice 'impreun'a cu tehnicile de interac'tiune pe care utilizatorul le are la dispozi'tie. Prima impresie pe care un utilizator 'si-o face despre o aplica'tie este puternic influen'tat'a de interfa'ta grafic'a a acestuia. Programele profesionale sunt atent ingrijite 'si dezvoltate, lucru care se vede at'at prin performan'tele acestora, c'at 'si prin {\ia}nfa'ti'sarea lor.

\medskip

Calitatea interfe'tei determin'a satisfac'tia consumatorului, iar ast'azi, c{\ia}nd pia'ta software este 'in plin'a dezvoltare iar competi'tia este pretutindeni, poate face diferen'ta dintre un produs de success sau unul e'suat. O interfa't'a eficient'a trebuie s'a fie 'in acela'si timp intuitiv'a, ergonomic'a, placut'a vizual iar 'in acela'si timp s'a expun'a toate capacit'a'tile produsului software.

\medskip

O interfa't'a intuitiv'a se remarc'a prin u'surin'ta cu care utilizatorii 'inva't'a s'a o foloseasc'a. Toate ac'tuinile posibile 'intr-o anumit'a situa'tie trebuie sa fie semnalate clar si s'a fie la 'indem{\ia}na utilizatorului. O astfel de interfa't'a nu las'a loc pentru interpretare. Comunic'area mesajelor se face clar, prin utilizarea de simboluri familiare 'si culori sugestive. Au fost sugerate moduri non-comformiste de testare a gradului de intuitivitate a unei interfe'te grafice precum: testarea aplica'tiei de c'atre persoane 'in v{\ia}rst'a sau complet nefamiliare cu sistemul sau 'stergerea complet'a a textului pentru a testa mesajul transmis de gama cromatic'a 'si a simbolurilor. Intuitivitatea unei aplica'tii determin'a c{\ia}t de u'sor se familiarizeaz'a utilizatorii noi cu produsul software, 'si c{\ia}t de repede ace'stia 'ii descoper'a func'tionalitatea. Pentru a face o bun'a prim'a impresie, un produs software trebuie s'a-'si fac'a utilizatorii comfortabili din primele momente.

\medskip

O interfa't'a ergonomic'a utilizeaz'a spa'tiul pe care 'il are la dispozi'tie 'intr-un mod inteligent. Unele din cele mai importante tr'as'aturi ale acestor interfe'te sunt:
\begin{itemize}
\item Afi'sarea selectiv'a a informa'tiei strict important'a pentru opera'tia pe care utilizatorul o efectueaz'a.
\item Utilizarea de simboluri sugestive 'in locul textelor.
\item Utilizarea de mesaje scurte dar concise pentru a minimiza spa'tiul utilizat 'si timpul de citire.
\item Utilizarea spa'tiului propor'tional cu interesul utilizatorului. 'In acest fel p'ar'tile importante ale aplica'tiei cu care utilizatorul interac'tioneaz'a des primesc mai mult spa'tiu dec'at cele rar utilizate.
\item Interpretarea corect'a a ac'tiunilor utilizatorului. Aceasta caracteristic'a este important'a pentru aplica'tiile mobile 'in special deoarece utilizeaz'a dispozitive de interac'tiune precum touchscreen-urile ce au un nivel de acurate'te foarte sc'azut.
\end{itemize}

O interfa't'a pl'acut'a vizual utilizeaz'a simboluri familiare, o gam'a cromatic'a uniform'a 'si estetic'a, f'ar'a a for'ta ochii 'si vederea utilizatorului. Aici men'tion'am calitatea fontului de a fi desenat 'si 'in'teles la dimensiunile utilizate de aplica'tie, calitatea icoanelor de a reprezenta simbolic informa'tie, contrastul 'si nu 'in ultimul r{\ia}nd calitatea anima'tiilor si a evit'arii artefactelor introduse de efectul de aliasing.

\medskip

'In concluzie, putem considera c'a este 'in avantajul nostru s'a consum'am cicluri adi'tionale de procesor 'si spa'tiu de memorie pentru a 'imbun'at'a'ti experien'ta utilizatorilor, deoarece productivitatea 'si satisfac'tia acestora dep'a'se'ste cu mult costurile modeste ale acestor resurse.

\section{Limbajul C++}

La momentul scrierii acestui proiect, C++ este unul dintre cele mai utilizate limbaje de programare. 'Incep{\ia}nd cu anul 1998 C++ este un limbaj standardizat, ceea ce 'inseamn'a c'a toate compilatoarele trebuie s'a implementeze acelea'si tr'as'aturi, s'a ofere acelea'si func'tionalita'ti 'si s'a accepte acelea'si programe care, o dat'a compilate, s'a se comporte identic. Popularitatea limbajului se datoreaz'a principiilor fundamentale pe care comitetul de standardizare le urm'are'ste \cite{cpplang}, 'si anume:

\begin{itemize}
\item \textit{S'a nu existe loc pentru un alt limbaj de programare mai jos de C++} ('in afar'a de limbajul de asamblare). Dac'a orice limbaj de programare ofer'a posibilitatea de a scrie programe mai eficiente, acel limbaj ar deveni limbajul de preferin't'a.
\item \textit{Nu pl'ate'sti ce nu folose'sti.} Orice tr'as'atur'a a limbajului sau o abstrac'tie fundamental'a trebuie proiectat'a s'a nu iroseasc'a nici-un ciclu de procesor 'in plus fa't'a de implement'arile alternative. Acesta se mai nume'ste \textit{principiul zero-overhead}.
\end{itemize}

'In plus fa't'a de particularit'a'tile sintaxei, a tipurilor de date 'si a specifica'tiilor fundamentale ale limbajului, standardul C++ a adoptat 'si specifica'tiile pentru o libr'arie standard numit'a Standard Template Library (STL). Aceast'a libr'arie con'tine algoritmi 'si structuri de date templatizate ce sunt dezvoltate, testate 'si 'intre'tinute de c'atre echipe de profesioni'sti. Libr'aria standard nu doar u'sureaz'a dezvoltarea de aplica'tii, dar garanteaz'a 'si calitatea, siguran'ta 'si performan'tele algoritmilor 'si a structurilor sale de date.

\medskip

Din nefericire standardul libajului C++ nu descrie o libr'arie pentru dezvoltarea de interfe'te vizuale. Acest lucru a fost men'tionat cu regret de c'atre creatorul limbajului \cite{cppessence} prin cuvintele: 
\textit{"We have no standard graphics and GUI, we have sort of maybe 25 competing frameworks which is just as bad as none."} 
Bjarne se refer'a f'ara 'indoial'a la API-urile sistemelor de operare 'si la libr'ariile care 'incearc'a s'a abstractizeze 'si uniformzeze aceste API-uri. Prin expresia \textit{25 competing frameworks which is just as bad as none} el pune accept pe faptul c'a incompatibilit'a'tile dintre aceste framework-uri 'ingreuneaz'a 'si 'incetinesc considerabil dezvoltarea de aplica'tii portabile.

\medskip

Ne afl'am 'in situa'tia in care unul dintre cele mai populare 'si performante limbaje de programare nu dispune de suportul necesar dezvoltarii de interfe'te vizuale moderne. Suntem obliga'ti s'a alegem dintre: a avea o singur'a platform'a 'tint'a, a porta aplica'tia pe mai multe platforme sau a folosi unul din framework-urile GUI portabile. Prima solu'tie este inacceptabil'a, deoarece aplica'tiile moderne sunt a'steptate s'a ruleze pe toate sistemele de operare majore, sau risc'a s'a piard'a 'in fa'ta competi'tiei. Cea de-a doua solu'tie este foarte costisitoare 'si 'int{\ia}rzie considerabil timpul de dezvoltare, ceea ce conduce la pierdere de oportunit'a'ti 'si poten'tiali clien'ti. Singura solu'tie viabil'a 'in momentul de fa't'a este utilizarea unui framework GUI portabil. Dintre acestea, cele mai mature sunt Qt 'si wxWidgets.

\section{Bibliotecile wxWidgets 'si QT}

Pentru a face o alegere 'intre cele dou'a framework-uri trebuie s'a le cunoa'stem tr'as'aturile, punctele forte 'si sl'abiciunile. O compara'tie de ansamblu 'intre cele dou'a se poate g'asi in tabelul \ref{comp}. Observ'am c'a cele dou'a framework-uri se aseam'an'a foarte mult, diferen'tele majore fiind platformele suportate, modul de implementare al componentelor vizuale 'si modul de licen'tiere.

\begin{center}
\begin{figure}[H]
    \centering
    \begin{tabular}{ |p{11cm}|p{2.5cm}|p{0.8cm}| }
        \hline
        \textbf{Tr'as'atur'a} & \textbf{wxWidgets} & \textbf{Qt} \\
        \hline
        Au sursele deschise (open-source)                                           & Da & Da \\
        Ofer'a componente vizuale fundamentale (butoane, etc.)                      & Da & Da \\
        Abstractizeaz'a evenimentele sistemului de operare                          & Da & Da \\
        Ofer'a posibilitatea extinderii prin noi componente                         & Da & Da \\
        Ofer'a suport pentru threading, networking                                  & Da & Da \\
        Ofer'a suport pentru grafic'a 2D                                            & Da & Da \\
        Permite posibilitatea schimb'arii aspectului unei componente                & Nu & Da \\
        Implementeaz'a componentele vizuale nativ                                   & Da & Nu \\
        Suport'a sistemele de operare desktop                                       & Da & Da \\
        Suport'a sistemele de operare mobile                                        & Nu & Da \\
        Permite utilizarea gratis pentru aplica'tii comerciale                      & Da & Nu \\
        Permite gratis linkuire static'a                                            & Da & Nu \\
        \hline
    \end{tabular}
    \caption{Compara'tie 'intre framework-urile GUI wxWidgets 'si Qt}
    \label{comp}
\end{figure}
\end{center}

'In ce priveste modul de implementare al componentelor vizuale, ambele libr'arii au calit'a'ti 'si defecte. Qt implementea'za toate componentele vizuale intern, folosind primitive de desenare 'si evenimentele primite de la sistemul de operare. Acest lucru face posibil'a stilizarea componentelor, 'si ofer'a aplica'tiilor o interfa't'a uniform'a pe toate platformele. wxWidgets implementeaz'a doar un layer superficial peste componentele native ale sistemului de operare. Acest lucru impiedic'a stilizarea componentelor dar ofer'a aplica'tiilor o interfa't'a nativ'a platformei pe care este rulat'a.

Din punct de vedere al licen'tei, wxWidgets este mult mai permisiv, oferind posibilitatea utiliz'arii libr'ariei in orice fel de aplica'tie (comercial'a sau nu) prin linkuire static'a sau dinamic'a. Acesta este un avantaj fa't'a de Qt care necesit'a o licen'ta pl'atit'a pentru aplica'tiile comerciale, iar pentru cele necomerciale necesit'a redistribuirea libr'ariilor dinamice de dimensiuni foarte mari.

De'si printre platformele suportate de Qt se num'ar'a 'si cele mobile, libr'aria wxWidgets urm'are'ste activ dezvoltarea in aceast'a direc'tie. Portarea pe platforma Android face parte din proiectele Google Summer of Code 2014\footnote{http://wxwidgets.org/news/2014/04/accepted-proposals-for-gsoc-2014/}, iar portarea pe telefoanele iOS fiind 'in dezvoltare.

Framework-ul Qt este sus'tinut financiar de organiza'tii interna'tionale 'si este mai bine dezvoltat dec{\ia}t wxWidgets. Din p'acate, pentru multe aplic'a'tii, licen'ta comercial'a a framework-ului Qt prezint'a un obstacol imposibil de dep'a'sit. Pentru aceste aplica'tii, wxWidgets este singura alternativ'a viabil'a.

% Putem contura aici o paralela intre o componenta hardware si o aplicatie software. Pentru a putea fi folosita, o componenta harwdare necesita o interfata software (driver) prin care sa comunice cu sistemul de operare. Aceasta interfata trebuie sa expuna intr-un mod cat mai eficient toate capacitatile componente hardware. Daca interfata este incompleta sau ineficienta, componenta hardware este prost utilizata si se comporta ineficient. Acest lucru determina formarea unei imagini gresite asupra performantelor hardware. Intr-un mod similar, performanta unei aplicatii software este limitata de interfata prin care utilizatorul o foloseste. 

% \medskip

% Putem considera deci ca este in avantajul nostru sa consumam cicluri aditionale de procesor si spatiu de memorie pentru a imbunatati experienta utilizatorilor, deoarece productivitatea si satisfactia acestora depaseste cu mult costurile modeste ale acestor resurse.

% \section{Domeniul proiectului}

% La momentul scrierii acestui proiect, C++ este unul dintre cele mai utilizate limbaje de programare datorita metodelor eficiente de abstractizare care nu sacrifica performanta si granularitatea cu care dezvoltatorul se poate apropia de nivelul hardware. Fiind un limbaj compilat, performantele acestuia sunt aproape de ne-rivalizat. LimDin nefericire, standardul libajului C++ nu descrie o librarie pentru construirea de interfete vizuale, acest lucru fiind posibil doar utilizand API-urile puse la dispozitie de sistemele de operare pe care aplicatia va rula. Aceste API-uri sunt complexe si incompatibile din punct de vedere al aspectelor suportate. Pentru a scrie aplicatii portabile, este deci necesara implementarea separata a componentelor de UI pentru fiecare platforma.

% \section{Solu'tia propus'a}

% \medskip
% % Din fericire, acest lucru este deja realizat de cateva librarii.

% Librariile wxWidgets si Qt sunt singurele librarii portabile care ofera un API uniform pentru toate platformele. Acestea sunt suficient de mature si complete pentru a fi mentionate in cartea de referinta a limbajului C++ "The C++ Programming Language" scrisa de Bjarne Stroustrup. Printre diferentele dintre librarii se numara gradul de maturitate, Qt fiind mult mai bine dezvoltata si licenta oferita, Qt necesitand o licenta comerciala pentru linkuirea statica. Asemanarile constau in portabilitate si paradigma OOP folosita de ambele librarii.

% \medskip

% O alta diferenta dintre cele doua librarii este gradul de stilizare al componentelor. Libraria Qt ofera acces usor la procesul de randare prin metode virtuale. Mai mult, Qt ofera posibilitatea stilizarii componentelor folosind un subset al limbajului CSS. Libraria wxWidgets prefera sa delege toata responsabilitatea randarii sistemului de operare.

% \section{Problema}

% \section{Solu'tia propus'a}
