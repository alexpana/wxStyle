\chapter{Studiu Bibliografic}
\pagestyle{headings}

'In cadrul activit'a'tii de studiu bibliografic, am apelat la c'ar'ti de referin't'a 'in domeniul graficii pe calculator, a activit'a'tii de proiectare 'si dezvoltare a interfe'te'lor vizuale dar 'si a utiliz'arii eficiente a tehnologiilor alese pentru dezvoltarea proiectului.

\section{Grafica pe calculator}

\subsubsection{Computer Graphics: Principles and Practice Second Edition}

Probabil una dintre cele mai influente c'ar'ti introductive ale literaturii de specialitate, \emph{Computer Graphics: Principles and Practice Second Edition}\cite{computergraphics} a fost publicat'a pentru prima dat'a 'in anul 1995. Edi'tia a doua, introduce elemente de grafic'a 2D si 3D raster, grafic'a vectorial'a, interfe'te utilizator, anti-aliasing, algoritmi avansa'ti de desenare 'si o introducere 'in anima'tie.

\medskip

Trebuie s'a 'tinem cont de contextul 'in care cartea a fost scris'a, mai specific perioada anilor 95, c{\ia}nd calculatoarele personale asistau la o revolu'tie digital'a a tehnologiei 'si se putea vorbi pentru prima dat'a de no'tiuni precum interfe'te utilizator. Din acest motiv, capitolele \emph{Chapter 8: Input Devices, Interaction Techniques, and Interaction Tasks}, \emph{Chapter 9: Dialogue Design} 'si \emph{Chapter 10: User Interface Software} prezint'a fundamentele necesare proiect'arii 'si implement'arii unei arhitecturi pentru interfe'te utilizator. Autorii prezint'a conceptele de interac'tiune cu utilizatorul prin tehnici 'si task-uri de interac'tiune 'si modul de abstractizare al acestor activit'a'ti prin metafore.

\medskip

Capitolele \emph{Chapter 3: Basic Raster Graphics Algorithms for Drawing 2D Primitives} 'si \emph{Chapter 5: Geometrical Transformation} ofer'a informa'tia necesar'a implement'arii algoritmilor pentru desenare 'in dou'a dimensiuni. Pentru scopul acestui proiect, biblioteca wxWidgets 'impreun'a cu sistemul de operare ofer'a suportul necesar 'si suficient al primitivelor de desenare, iar aceste capitole sunt doar pentru a dezvolta o mai bun'a 'in'telegere a detaliilor.

\section{Proiectarea 'si dezvoltarea de interfe'te utilizator vizuale}

Despre proiectarea 'si dezvoltarea interfe'telor utilizat'or vizuale au fost scrise numeroase c'ar'ti, articole, studii, etc. Pentru proiectul de fa't'a, cele mai am ales trei dintre cele mai cunoscute

\subsubsection{The Old New Thing: Practical Development Throughout the Evolution of Windows}

\emph{The Old New Thing: Practical Development Throughout the Evolution of Windows}\cite{theoldnewthing} a fost scris'a 'in anul 2007 de c'atre Raymond Chen, unul dintre cei mai vechi dezvoltatori Microsoft. Autorul  'inc'a de la apari'tia primelor versiuni de Windows. Cartea sa este 'imp'ar'tit'a in scurte recolec'tii 'si povestiri care ofer'a un context deciziilor luate de echipa Microsoft la proiectarea interfe'telor vizuale pentru sistemele de operare Windows.

\medskip

Raymond Chen poveste'ste dificult'a'tile utilizatorilor sistemului de operare, frustr'arile 'si reac'tiile acestora 'in momentul confrunt'arii cu interfa'ta neintuitiv'a 'si necizelat'a a primelor versiuni de Windows. Autorul explic'a 'si deciziile luate de echipa responsabil'a pentru dezvoltarea interfe'tei grafice a sistemului de operare pentru rezolva problemele utilizatorilor. Cartea este relevant'a proiectului deoarece prezint'a modurile 'in care utilizatorii interactioneaz'a cu un sistem software, care sunt obieceiurile 'si instinctele lor 'si ce solu'tii se pot adopta pentru a asigura o interac'tiune c{\ia}t mai u'soar'a cu sistemul. Aceste solu'tii necesit'a un set de unelte pentru a fi implementate, printre care se num'ar'a si biblioteca wxStyle.

\subsubsection{Designing Interactive Systems: a comprehensive guide to HCI and interaction design}

Una dintre cele mai renumite c'ar'ti din domeniuni interac'tiunii om - calculator, cartea este 'imp'ar'tit'a 'in studii de caz extrase din experien'ta autorului. Fiecare studiu prezint'a o provocare 'in ceea ce prive'ste arhitectura 'si implementarea unui sistem interactiv. Printre capitole care parcurg tehnicile esentiale in dezvoltarea aplica'tiilor interactive, autorul atinge 'si subiecte precum design-ul paginilor web 'si experien'ta utilizatorului. Prin experien't'a utilizator 'in'telegem emo'tiile 'si atitudinea utilizatorului fa't'a de sistem. La fel ca 'in cazul c'ar'tii lui Raymond Chen, aceast'a carte ajut'a la modelarea bibliotecii wxStyle prin definirea scopului si problemelor pe care biblioteca trebuie s'a le rezolve.

\subsubsection{Designing the user interface: strategies for effective human-computer interaction}

Aceast'a carte ofer'a o introducere aprofundat'a 'in domeniul interac'tiunii om - calculator. O echip'a de autori 'i'si aduc experien'ta prezent{\ia}nd cele mai bune practici 'si principii moderne 'in dezvoltarea aplica'tiilor dinamice. Cartea atac'a probleme contemporane precum experien'ta utilizatorului, re'tele de socializare, con'tinut generat de utilizatori, precum 'si probleme de securitate 'si spam. Cartea este dedicat'a 'in special platformelor mobile 'si web, dar principiile 'si practicile prezentate se aplic'a 'in egal'a m'asura aplica'tiilor desktop. Biblioteca wxStyle are rolul de a face posibile aceste practici 'in contextul de dezvoltare wxWidgets.

\subsubsection{Alte biblioteci 'si arhitecturi similare}

'In plus fa't'a de aceste c'ar'ti, am studiat bibliotecile \emph{Qt}, \emph{CEGUI (Crazy Eddie`s GUI System)}, libr'aria \emph{Swing} a limbajului Java 'si limbajele HTML 'si CSS. Toate aceste libr'arii sau limbaje au 'in comun nu doar suportul flexibil pentru construirea de obiecte de interfa't'a stilizabile, dar si separarea clar'a 'intre descrierea stilului utilizat la prezentarea unui obiect si implementarea acestuia. Aceste arhitecturi ofer'a o referin't'a 'si o funda'tie solid'a pentru deciziile luate 'in acest proiect.

\section{Tehnologii}

Aceast'a sec'tie cuprinde bibliografia referitoare at{\ia}t la tehnologiile utilizate de biblioteca wxStyle, c{\ia}t 'si exemple de tehnologii ce pot fi implementate folosind aceasta bibliotec'a.

\subsubsection{Cross-Platform GUI Programming with wxWidgets}

Biblioteca wxStyle este implementa'ta deasupra bibliotecii wxWidgets, deci este necesar'a o 'in'telegere profund'a a arhitecturii 'si implement'arii acesteia din urm'a.

\medskip

Singura carte ce trateaz'a biblioteca wxWidgets este cea scris'a de autorul original al bibliotecii Julian Smart, 'si este intitulat'a \emph{Cross-Platform GUI Programming with wxWidgets}\cite{wxwidgetsguide}. Aceast'a carte 'incepe prin a prezenta scopul original al bibliotecii: acela de a oferii dezvoltatorilor de aplica'tii portabile pentru ma'sini desktop sau mobile un set de unelte necesare dezvolt'arii de aplica'tii vizuale. Capitolele ce urmeaz'a prezint'a toate aspectele bibliotecii, 'incep{\ia}nd cu funda'tiile dezvolt'arii de aplica'tii vizuale prin procesarea de evenimente si utilizarea de obiecte de interfa't'a, apoi continu{\ia}nd cu capitole ce trateaza interna'tionalizarea, scrierea de programe paralele 'si utilizarea socket-urilor pentru a scrie programe ce interac'tioneaz'a cu re'teaua.

\medskip

Capitolele relevante acestei lucr'ari sunt \emph{Chapter 3 Event Handling}, \emph{Chapter 4 Window Basics}, \emph{Chapter 5 Drawing and Printing} 'si \emph{Chapter 6 Handling Input}. Capitolul 3 prezint'a sistemul de procesare a evenimentelor implementat 'in biblioteca wxWidgets. Acesta cuprinde utilizarea mecanismului de capturare 'si procesare de evenimente at{\ia}t dinamic c{\ia}t 'si static, generarea de evenimente 'si construirea unor tipuri noi de evenimente. Capitolul 4 ofera o privire de ansamblu asupra anatomiei unei ferestre. Informa'tiile din acest capitol sunt folositoare pentru implementarea propriei ferestre stilizabile. Capitolul 5 cuprinde aproape toate detaliile necesare implement'arii procesului de prezentare prin descrierea uneltelor de desenare 2D 'si a modului de utilizare al acestora. Capitolul 6 prezint'a modul de tratare al evenimentelor generate de interac'tiunea utilizatorului cu aplica'tia. Aceste evenimente de nivel jos (dispozitiv mouse 'si tastatur'a) pot fi folosite pentru a implementa logica fiec'arui obiect de interfa't'a.

Aceast'a carte este completat'a de wiki-ul\footnote{http://wiki.wxwidgets.org/} bibliotecii wxWidgets, 'impreun'a cu documenta'tia oficial'a\footnote{http://docs.wxwidgets.org/3.0/} a API-ului 'si forumul\footnote{http://forums.wxwidgets.org/} comunit'a'tii.

\subsubsection{The C++ Programming Language}

Cartea de referin'ta a limbajului C++ se numeste \emph{The C++ Programming Language}, iar edi'tia a patra a fost publicat'a de c'atre autorul limbajului Bjarne Stroustrup 'in anul 2013. Aceast'a edi'tie este relevan'ta pentru standardul C++11 al limbajului 'si con'tine informa'tii despre toate aspectele limbajului de programare de la tipuri de date primitive, operatori, expresii 'si instruc'tiuni p{\ia}n'a la mecanisme avansate de abstractizare precum clase 'si continutul bibliotecii standard.

\medskip

Aceast'a carte este esen'tial'a pentru a 'in'telege limbajul C++ 'in 'intregime 'si serve'ste rolul de referin't'a pentru orice aspect al limbajului. Cartea poate fi considerat'a o versiune mai u'sor de citit 'si 'in'teles, dar mai restr{\ia}ns'a din punct de vedere al con'tinutului, a standardului C++.

\subsubsection{Exceptional C++}

Una din cele mai renumite c'ar'ti ce trateaz'a subiectul utiliz'arii corecte a limbajului C++ este \emph{Exceptional C++}\cite{exceptional_cpp} scris'a de Herb Sutter. Autorul face parte din comisia de standardizare a limbajului 'si este o figur'a renumit'a pentru prezent'arile sale 'in cadrul conferin'telor de specialitate, pentru c'ar'tile considerate de c'ap'at{\ia}i pentru orice programator C++ 'si pentru articolele sale celebre despre limbajul C++ intitulate colectiv \emph{Guru of the Week}.

\medskip

Cartea este structurat'a ca o serie de 'intreb'ari 'si r'aspunsuri pe tema utiliz'arii corecte a limbajului, 'in special securitate, tratarea excep'tiilor 'si dezvoltarea de arhitecturi stabile 'si eficiente prin utilizarea corect'a a tr'as'aturilor limbajului 'si a libr'ariei standard.

\subsubsection{Effective C++, More Effective C++ 'si Effective STL}

Aceste trei c'ar'ti\cite{effective_cpp}\cite{more_effective_cpp}\cite{effective_stl} scrise de autorul Scott Meyers fac parte din biblioteca obligatorie a programatorului C++, 'impreun'a cu c'ar'tile lui Herb Sutter. Aceste c'ar'ti trateaz'a utilizarea corect'a a limbajului 'si a tr'as'aturilor sale mai mult sau mai pu'tin obscure. C'ar'tile sunt organizate in item-uri, fiecare prezent{\ia}nd o regul'a general valabil'a care poate fi aplicat'a 'in orice situa'tie pentru a garanta corectitudinea unui program.

\medskip

Primele dou'a c'ar'ti se preocup'a de utilizarea corect'a a nucleului limbajului 'si metodelor sale de abstractizare, de utilizarea corect'a a claselor, de suprascrierea corect'a a operatorilor 'si de evitarea gre'selilor uzuale. Scott Meyers acoper'a 'si problema administr'arii memoriei prin prezentarea metodei RAII 'si a pointer-ilor inteligen'ti din biblioteca standard. Cartea din urm'a trateaz'a utilizarea corect'a a libr'ariei standard printre care cunoasterea diverselor tipuri de iteratori 'si familiarizarea cu colec'tiile 'si preforman'tele acestora.

\subsubsection{Large-Scale C++ Software Design}

Cartea \emph{Large-Scale C++ Software Design}\cite{largescalecpp}, scris'a de autorul John Lakos 'in anul 1997 adreseaz'a problemele des 'int{\ia}lnite 'in dezvoltarea aplica'tiilor de dimensiuni extinse folosind limbajul C++. Deoarece biblioteca wxStyle con'tine un num'ar relativ ridicat de fi'siere surs'a (aproximativ 70 la momentul scrierii acestei documenta'tii) care implementeaz'a componente distincte (obiecte de interfa't'a, instruc'tiuni de desenare, obiecte de prezentare, analizator de fi'siere de stil) ce formeaz'a diverse rela'tii de dependin't'a (cum ar fi rela'tiile de mo'stenire 'si agregare) putem considera proiectul suficient de extins pentru a beneficia de informa'tiile prezente 'in aceast'a carte.

\medskip

Printre regulile fundamentale recomandante de John Lakos se num'ar'a evitarea expunerii implement'arilor prin metoda pointerilor opaci sau pointerilor c'atre implementare (PIMPL idiom), minimizarea dependin'telor 'intre module folosind declara'tii 'inainte (forward declarations), evitarea categoric'a a dependin'telor ciclice 'si testarea independent'a a modulelor. Alte informa'tii detaliate despre structura limbajului ajut'a la dezvoltarea de aplica'tii sigure (prin evitarea greselilor subtile), usor de 'intre'tinut 'si de extins (prin structurarea aplica'tiilor 'in arhitecturi flexibile).

\subsubsection{User interface contrast filter}

Prin adaugarea suportului pentru stilizare, biblioteca wxStyle ofer'a dezvoltatorilor posibilitatea de a schimba dinamic, 'in timpul rul'arii unei aplica'tii inf'a'ti'sarea obiectelor de interfa't'a. Aceast'a nou'a posibilitate deschide por'tile unei genera'tii noi de aplica'tii care se pot adapta la factori externi precum luminozitatea sau temperatura mediului.

\medskip

O astfel de tehnologie este patentat'a de Apple\cite{uicontrast} 'si specific'a o metod'a prin care aplica'tiile vizuale ce ruleaz'a pe dispozitive mobile 'i'si pot schimba contrastul pentru a se adapta la condi'tii de vizibilitate redus'a determinate de luminozitatea extern'a. De'si aceast'a tehnologie este ap'arat'a de legile privind dreptul de autor 'si este curent de'tinut'a de Apple, ea este un exemplu deosebit de relevant pentru scopul existen'tei bibliotecii wxStyle. Mai precis, aplica'tiile moderne au nevoie de mecanisme u'sor de utilizat pentru manipularea dinamic'a a interfe'telor vizuale.